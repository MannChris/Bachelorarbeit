%     WORK IN PROGRESS %
\documentclass[11pt,a4paper,twoside, draft]{article}%draft wieder rausnehmen
%\documentclass[11pt,a4paper,twoside]{article}
\usepackage[a4paper, bindingoffset=0.5cm, hmargin={2.5cm, 2.5cm},vmargin={2.5cm, 2.5cm}]{geometry}
\usepackage[T1]{fontenc} 
\usepackage[utf8]{inputenc} 
\usepackage[ngerman,english]{babel} 
\usepackage{mathptmx} 
\usepackage{mathtools}
\usepackage{amsmath}
\usepackage{amssymb}
\usepackage{bbm}
\usepackage{latexsym}
\usepackage{nicefrac}
\usepackage{comment}
\usepackage{fp}
\usepackage{cite}
\usepackage{url}
\usepackage{float}
\usepackage{caption}
%\usepackage[ngerman, num]{isodate}
\usepackage[ngerman]{datetime}
\newtheorem{Tm}{Satz}[section]
\newtheorem{Tms}{Satz}[subsection]
\newtheorem{Lm}[Tm]{Lemma}
\newtheorem{Df}[Tm]{Definition}
\newtheorem{Lms}[Tms]{Lemma}
\newtheorem{Dfs}[Tms]{Definition}
\newtheorem{Bsps}[Tms]{Beispiel}
\newtheorem{Bsp}[Tm]{Beispiel}
%Abbildungen sollen Abb. genannt werden und nicht Figure
\newdateformat{myformat}{\THEDAY{. }\monthnamengerman[\THEMONTH], \THEYEAR}
\addto{\captionsngerman}{%
  \renewcommand*{\contentsname}{Inhalt}
  \renewcommand*{\listfigurename}{Abbildungen}
  \renewcommand*{\figurename}{Abb.}
}
\addto{\captionsenglish}{%
  \renewcommand*{\contentsname}{Inhalt}
  \renewcommand*{\listfigurename}{Abbildungen}
  \renewcommand*{\figurename}{Abb.}
}

%\ifpdf %%Einbindung von Grafiken mittels \includegraphics{datei}
	\usepackage[pdftex]{graphicx} %%Grafiken in pdfLaTeX
%\else
%	\usepackage[dvips]{graphicx} %%Grafiken und normales LaTeX
%\fi
%\usepackage[hang,tight,raggedright]{subfigure} %%Teilabbildungen in einer Abbildung
%\usepackage{pst-all} %%PSTricks - nicht verwendbar mit pdfLaTeX
\usepackage{fancyhdr}
\pagestyle{fancy}
%% Zeilenabstand %%%%%%%%%%%%%%%%%%%%%%%%%%%%%%%%%%%%%%%%%%%%
%\usepackage{setspace}
%\singlespacing        %% 1-zeilig (Standard)
%\onehalfspacing       %% 1,5-zeilig
%\doublespacing        %% 2-zeilig

%absaetze
\parindent0cm
\parskip0.2cm

%\usepackage[pdftex,bookmarks=true,colorlinks=true,
%        linkcolor=black,
%        citecolor=black,
%        filecolor=black,
%        pagecolor=black,
%        urlcolor=black,
%        bookmarks=true,
%        bookmarksopen=true,
%        bookmarksopenlevel=3,
%        plainpages=false,
%        pdfpagelabels=true]{hyperref}


% Title Page

\usepackage{todonotes}
\usepackage{mwe}

\begin{document}
%% Abschnittsueberschriften auf rechter Seite(odd) links
%% - Abschnitt - Seitenzahlen aussen
\lhead[\fancyplain{}{\thepage}]{\fancyplain{}{\rightmark}}
%% Kapitelueberschriften auf linker Seite(even) rechts
%% - Kapitel - Seitenzahlen aussen
\rhead[\fancyplain{}{\leftmark}]{\fancyplain{}{\thepage}}
\cfoot{}
\pagenumbering{roman}
\begin{titlepage}
\begin{center}
\vspace*{0.65cm}
\huge\pagenumbering{arabic}
\setcounter{page}{1}
\hspace*{-0.73cm}
\textbf{Das Matrix-Tree-Theorem}\\
\vspace*{2.2cm}

\begin{figure}[h]
	\begin{center}
		\hspace*{-0.73cm}
		\includegraphics{lmu_siegel}
	\end{center}
\end{figure}
%%
\vspace*{0.5cm}
\Large
\hspace*{-0.73cm}
\hspace*{-0.5cm}Bachelorarbeit der Fakultät für Mathematik\\
\vspace*{0.1cm}
\hspace*{-0.73cm}
\hspace*{-0.5cm}der\\
\vspace*{0.1cm}
\hspace*{-0.73cm}
\hspace*{-0.5cm}Ludwig-Maximilians-Universität München\\
%%
%\vspace*{3.7cm}
\vspace*{3.5cm}
\large
\hspace*{-0.73cm}
\hspace*{-0.8cm} vorgelegt von\\
\vspace*{0.1cm}
\hspace*{-0.73cm}
\hspace*{-0.7cm}\Large \textbf{Christopher Mann}\\
\vspace*{0.15cm}
\large
\hspace*{-0.73cm}
\hspace*{-0.8cm} geboren in Freising\\
\vspace*{1.5cm}
\vspace*{0.4cm}Betreuer: \textbf{Prof. Dr. Konstantinos Panagiotou}\\
\vspace*{1cm}
\hspace*{-0.73cm}
\hspace*{-0.43cm}München, den \myformat\today\\
\end{center}
\end{titlepage}


%
%%%%%%%%%%%%%%%%%%%%%%%%%%%%%%%%%%%%%%%%%%%%%%%%%%%%%%%%%%%%%%%%%%%%%%
%
\clearpage{\pagestyle{empty}\cleardoublepage}
%
%%%%%%%%%%%%%%%%%%%%%%%%%%%%%%%%%%%%%%%%%%%%%%%%%%%%%%%%%%%%%%%%%%%%%%


%\thispagestyle{empty}

\cleardoublepage

\pagestyle{plain}
\pagenumbering{arabic} 
\setcounter{page}{1}

\tableofcontents
\cleardoublepage

\section{Einleitung}
%Graphentheorie spielt in allen MINT-Fächern eine Rolle. 
%Sie ist daher ein besonders zukunftsweisender Zweig der Mathematik. 
%Diese Arbeit beschäftigt sich mit sogenannten Matrix-Tree-Theorems.  
Schon 1847 schrieb Kirchhoff in seiner Arbeit "Ueber[sic!] die Auflösung der Gleichungen, auf welche man bei der Untersuchung der linearen Vertheilung[sic!] galvanischer Ströme geführt wird" \cite{kirchhoff_1847} als erster implizit über den Zusammenhang zwischen Matrizen und der Anzahl von Spannbäumen von Graphen - explizit über Systeme von beliebig miteinander verbundenen Drähten und Gleichungen für die elektrischen Stromstärken in diesem System.
In der Graphentheorie sind Matrix-Tree-Theorems neben anderen Methoden, wie Prüfer-codes, beliebte Werkzeuge um die Anzahl der Spannbäume eines Graphen zu ermitteln.
Das wohl berühmteste, nach Kirchhoff benannte, Matrix-Tree-Theorem wird oft als "das Matrix-Tree-Theorem" bezeichnet, wobei es auch noch andere Versionen, wie Tuttes Matrix-Tree-Theorem für gerichtete Multigraphen, dessen Beweis auch Teil dieser Arbeit sein wird, gibt.
Aber auch außerhalb der reinen Mathematik und der Theorie über elektrische Schaltkreise finden Matrix-Tree-Theorems ihre Anwendung.\\
So wird das Matrix-Tree-Theorem zum Beispiel in der Quantenphysik genutzt \cite{giovannetti_severini_2013}.
In der Chemie gibt es einen Zusammenhang zwischen Spannbäumen und der Complexität von Molekülen \cite{janezic_2015}.
Auch in der Informatik kann man Matrix-Tree-Theorems zum Beispiel dafür benutzen, die Anzahl von Spannbäumen von Netzwerken - die man als Graphen betrachen kann - zu berechnen, um dann Rück\-schlüsse auf die Stabilität dieser Netzwerke zu ziehen \cite{yakoubi_2019}.
Das Anwendungsspektrum von Matrix-Tree-Theorems ist also vielseitig.
%In dieser Arbeit werden wir uns zwei Matrix-Tree-Theorems erschließen und damit im Anschluss die Anzahl der Spannbäume von Graphen einiger Klassen bestimmen.
In dieser Arbeit werden wir zuerst ein Matrix-Tree-Theorem für gerichtete Multigraphen beweisen, wobei wir uns an \cite{bang-jensen_2009} halten. Daraus werden wir dann Kirchhoffs Matrix-Tree-Theorem schließen. In Kapitel "Warm-up und interessante Identität" werden wir dann die Anzahl der Spannbäume von Pfad- und Kreisgraphen berechnen; auf diesem Weg sehen wir eine interessante Identität des Sinus. Danach werden wir den Satz von Cayley beweisen; dabei imitieren wir \cite{Lau_2004}.
Als nächste Graphenklasse betrachten wir dann vollständige multipartite Graphen; dort modifizieren \cite{austin_1960} für unsere Zwecke. Wie \cite{bogdanowicz_2008} zeigen wir danach einen Zusammenhang zwischen den berühmten Fibonaccizahlen und der Anzahl der Spannbäume von Fächer-Graphen, die uns im darauffolgenden Kapitel zu Rad-Graphen nützlich sein wird; an dieser Stelle werden wir uns von \cite{sedlacek_1970} inspirieren lassen. 
Mit einem Beweis, der \cite{wang_yang_1984} ähnlich ist, berechnen wir dann die Anzahl der Spannbäume in zirkulären Graphen. Wir runden die Arbeit dann mit einem Kapitel über kartesische Produkte von Graphen ab. 


\section{Grundlegende Definitionen und Notationen}
Wir beginnen damit, ein paar wichtige Begriffe und Notationen einzuführen, die wir später häufiger benutzen werden.
In einem Matrix-Tree-Theorem wird immer ein Zusammenhang zwischen bestimmten Matrizen und den Spannbäumen eines Graphen beschrieben. Daher bieten sich die folgenden zwei Definitionen an, wobei wir eine für ungerichtete und die andere für gerichtete Graphen verwenden werden.
\begin{Df}\textbf{Laplacematrix}\\
Für einen Graphen $G=(V,E)$ indizieren wir die Zeilen und Spalten einer Matrix mit den Knoten von G. Die Matrix mit den Einträgen
$$
l_{ij}=
\begin{cases}
 d(i),\, \, falls\,\, i=j,\\
 (-1), \, \, falls \,\, ij \in E,\\
 0, \,\, sonst.
\end{cases}
$$
nennen wir die Laplacematrix $L(G)$ des Graphen $G$.
\end{Df}
Für einen Knoten $i$ bezeichnen wir mit $d^{-}(i)$ seinen Ausgangsgrad.
\begin{Df}\textbf{Kirchhoffmatrix}\\
 Für einen gerichteten Multigraphen $D$ mit Knoten $1,..,n$ definieren wir die Kirchhoffmatrix $K(D)$ wie folgt:\\
 $$
k_{ij}=
\begin{cases}
 d^{-}(i),\, \, falls\,\, i=j,\\
 (-m), \, \, wobei \,\, m \,\, die\, Anzahl\, der\, Kanten\, von\,\, i\,\, nach \,\,j\,\, ist.
\end{cases}
$$
Wobei wir wieder die Zeilen und Spalten mit den Knoten von $D$ indiziert haben.\\
Nachdem wir in Tuttes Matrix-Tree-Theorem eine Zeile und Spalte streichen müssen, bezeichnen wir mit $K_{\bar{i}}(D)$ die Matrix, die entsteht, wenn man aus $K(D)$ die $i$-te Zeile und Spalte löscht.
\end{Df}
Im Verlauf dieser Arbeit werden wir immer wieder die Anzahl der Spannbäume eines Graphen ausrechnen, daher verwenden wir $\mathit{k}(G)$ als die Anzahl der Spannbäume eines beliebigen Graphen $G$.
\todo[inline, color=yellow]{Stand jetzt werde ich die Definitionen als solche hervorheben, Notation jedoch im Fließtext unterbringen, passt das so?-JA}

\section{Das Matrix-Tree-Theorem}

Nachdem wir genügend Vorarbeit geleistet haben, können wir mit dem wichtigsten Teil dieser Arbeit anfangen, dem Beweis der Matrix-tree-theorems selbst. Wir beweisen zuerst eine Version für gerichtete Multigraphen, bevor wir uns der Version für ungerichtete Graphen als einem Spezialfall davon widmen.

\subsection{Tutte's Matrix-Tree-Theorem}
\begin{Tms}[Tutte's Matrix-Tree-Theorem]
Sei D ein gerichteter Multigraph mit Kirchoffmatrix K(D). Die Anzahl der out-branchings aus dem Knoten i ist gleich der det($K_{\bar{i}}(D)$).
\end{Tms}
\textbf{Beweis:} 
\todo[inline]{Beweis schreiben}

\subsection{Kirchhoff's Matrix-Tree-Theorem}
\begin{Tms}[Kirchoff ’s Matrix Tree Theorem]
Sei G ein ungerichteter Graph und $L_n$  die dazugehörige Laplacematrix. 
Dann gilt:
% Bis hier Text ohne Einrückung
\par
\begingroup
\leftskip=20pt% Parameter anpassen
\rightskip=20pt
\noindent %ab hier der Text, der eingerückt werden soll
(1) Die Anzahl der Spannbäume von G gleich einem beliebigen Kofaktor von $L_n$.\\
(2) Die Anazahl der Spannbäume von G ist gleich $\frac{1}{n}\lambda_1\ldots\lambda_{n-1}$, wobei $\lambda_1,\ldots,\lambda_{n-1}$ die Eigenwerte von $L_n$ sind, die ungleich null sind.
\par
\endgroup
% ab hier wieder Text ohne Einrückung
\end{Tms}
\textbf{Beweis:}
 



\section{Anzahl Spannbäume für bestimmte Graphenklassen}
Nachdem Kirchhoff's Matrix-Tree-Theorem nun bewiesen ist, werden wir damit im Folgenden Formeln für die Berechnung der Anzahl der Spannbäume für verschiedene Klassen von ungerichteten Graphen finden. Begegnen werden uns unter Anderem der vollständige Graph, multipartite Graphen, Räder und as Quadrat eines Kreises (Square of a cycle)).Dabei werden wir uns an der ein- oder anderen Stelle ein paar Eigenschaften bestimmter Matrizen, Determinanten, aber auch zum Beispiel von Chebychev-polynomen zunutze machen, da das Ausrechnen eines Kofaktors der Laplacematrix hier manchmal nicht der schnellste und intelligenteste Weg ist um ans Ziel zu kommen. 

\subsection{Vollständige Graphen - der Satz von Cayley}
%Als Einstieg soll der vollständige Graph mit n Knoten kurz $\,K_n\,$ dienen.\\%%%Wer hat das zuerst herausgefunden? Wo haben wir den Beweis her?Warum ist das interessant
\begin{Tms}[Satz von Cayley]
$K_n\,$ besitzt genau $\,n^{n-2}\,$ verschiedene Spannbäume.\\
\end{Tms}
\textbf{Beweis:}\\
Unser Beweis orientiert sich an~\cite{Lau_2004}. Wir wollen das Matrix-Tree-Theorem verwenden und betrach\-ten deshalb die Determinante der Matrix, die durch das Streichen der ersten Zeile und Spalte der Laplacematrix $\,L_n(K_n)\in M_n(\mathbb{Z})\,$ entsteht:
\begin{equation*}
det
\begin{pmatrix}
n-1&-1&\ldots&\ldots&\ldots&-1\\
-1&n-1&-1&\ldots&\ldots&-1\\
-1&-1&n-1&-1&\ldots&-1\\
\ldots&\ldots&\ldots&\ldots&\ldots&\ldots&\\
-1&\ldots&\ldots&\ldots&-1&n-1\\
\end{pmatrix}
\end{equation*}
Da sich die Determinante durch elementare Zeilen- und Spaltenoperationen nicht ändert, dürfen wir die erste Spalte von allen anderen subtrahieren und erhalten:
\begin{equation*}det
\begin{pmatrix}
n-1&-n&\ldots&\ldots&\ldots&-n\\
-1&n&0&\ldots&\ldots&0\\
-1&0&n&0&\ldots&0\\
\ldots&\ldots&\ldots&\ldots&\ldots&\ldots&\\
-1&0&\ldots&\ldots&0&n\\
\end{pmatrix}
\end{equation*}
Mit demselben Argument wie oben addieren wir zur ersten Zeile alle übrigen und es ergibt sich:
\begin{equation*}
det
\begin{pmatrix}
1&0&\ldots&\ldots&\ldots&0\\
-1&n&0&\ldots&\ldots&0\\
-1&0&n&0&\ldots&0\\
\ldots&\ldots&\ldots&\ldots&\ldots&\ldots&\\
-1&0&\ldots&\ldots&0&n\\
\end{pmatrix}
\end{equation*}
Wir berechnen den Wert dieser Determinante durch Entwicklung nach der ersten Zeile. \\
Weil die betrachtete Matrix eine $\,(n-1 \times n-1)$-Matrix ist, ist die Determinante gleich $\,n^{n-2}$.\; \\
Nach Kirchhoffs Matrix-Tree-Theorem ist genau das die Anzahl der Spannbäume des $\,K_n$.\; 
\begin{flushright} $\,\Box\,$ \end{flushright} 

\subsection{vollständige multipartite Graphen}
 
Als nächste Graphenklasse betrachten wir vollständige multipartite Graphen.
\todo[inline]{Notation}

\begin{Tms}
 Für die Anzahl der Spannbäume in einem vollständig m-partiten Graphen $K_{n_1,..,n_m}$ mit n Knoten gilt:\\
 $\mathit{k}(K_{n_1,..,n_m})=n^{m-2}\prod_{i=1}^{m}(n-n_1)^{n_i-1}$
\end{Tms}
\textbf{Beweis:}
Wir beweisen den Satz ähnlich wie Austin in ~\cite{austin_1960}, der ein äquivalentes Problem zu dem ebengenannten bewiesen hat.\\
Dazu werden wir im Geist dieser Arbeit Kirchhoffs Matrix-Tree-Theorem verwenden.\\
Zuerst werden wir bemerken, dass alle Laplace-Matrizen, die unseren Sachverhalt beschreiben, bei geschickter Nummerierung Blockmatrizen einer bestimmten Gestalt sind und Schlüsse über deren Kofaktoren  ziehen. Im nächsten Schritt werden wir dann einen beliebigen solchen Graphen auswählen und die entsprechenden Werte einsetzen. Mit Kirchhoffs Matrix-Tree-Theorem folgt dann der Satz.\\

\todo[inline]{Beweis fertigmachen}

\subsection{Cartesische Produkte von Graphen}
In diesem Teil zeigen wir, was im Bezug auf die Anzahl der Spannbäume geschieht, wenn man das kartesische Produkt von Graphen bildet. %Wie schaut so ein Graph dann überhaupt aus? 
\todo[inline, color=yellow]{Ich werde wahrscheinlich ein, zwei Beispiele davon zeigen, aber nicht mehr}
\begin{Lm}
eigenwerte kartesisches Produkt v. Graphen (Kronekersumme Aidb + idaB)
\todo[inline, color=red]{das ist eigentlich kein Lemma bleibt aber für den Moment so markiert}
\end{Lm}

\begin{Tms}
 Sei $G$ ein Graph mit $m$ Knoten und Eigenwerten $\mu_1(G),..,\mu_m(G)$ und $H$ ein Graph mit $n$ Knoten und Eigenwerten $\mu_1(H),..,\mu_n(H)$. \\
 Dann sind die Eigenwerte des kartesischen Produkts $G \times H$ genau $\mu_i(G)+\mu_j(H)$ mit $i \in \{ 1,..,m\}, j \in \{ 1,..,n\}$.
\end{Tms}
\todo[inline]{Beweis schreiben}


\subsection{$F_n$ (Fan)(Fächer?)}
Nun werden wir Fan-Graphen $F_n$, für $n \geq 1$betrachten. Diese entstehen wenn wir an einen Pfad-Graphen $P_{n}$ einen weiteren Knoten so ankleben, dass er mit allen übrigen Knoten adjazent ist. 
% Interessanterweise treffen wir hier auf die Fibonaccizahlen, blablabla
\todo[inline, color= red]{Achtung: Fn bei uns ist Fn+1 im Paper}
Wir wollen in diesem Kapital folgendes über die Anzahl der Spannbäume in Fan-Graphen zeigen:
\begin{Tms}
 %%Formel hier
 \label{ThmFn}
\end{Tms}
\textbf{Beweis:}
Diesmal halten wir uns an einen Beweis von Bogdanowicz ~\cite{bogdanowicz_2008}, wobei dieser $F_n$ leicht anders definiert.\\
Zuerst werden wir zeigen, dass ein Kofaktor der Laplacematrix von Fan-Graphen einer bestimmten Rekursion folgt und dann, dass $Fib(2n)$ der die gleiche Rekursionsvorschrift einhält; Mit Kirchhoffs Matrix-Tree-Theorem folgt dann der Satz.\\
Wir betrachten also zunächst die Laplacematrix von $F_n$; wir dürfen dazu die Knoten nummerieren wie wir wollen, also bekommen wir\\
\todo[inline]{matrix, zeilenumbrüche checken}
Wir brauchen einen beliebigen Kofaktor davon, deshalb streichen wir die letzte Zeile und Spalte und erhalten\\
\todo[inline]{matrix, zeilenumbrüche checken, An nennen}
Die Determinante dieser Matrix ist der gesuchte Kofaktor; wir benennen sie mit $a_n$.\\
Nun zeigen wir,dass die Folge $(a_n)_{n \in \mathbb{N}}$ der Rekursion $x^2-3x+1=0$ folgt, \\wobei $x$ den Shift-Operator $a_n = xa_{n-1}$ darstellt. \\
Wir entwickeln $A_n$ nach der ersten Reihe und erhalten $a_n = 2b_{n-1} - b_{n-2}$, wobei $b_i$ die Determinante der folgenden Hilfsmatrix ist:\\
\todo[inline]{matrix, zeilenumbrüche checken}
Entwickeln wir die Determinante dieser Matrix für $i=n$ ebenfalls nach der ersten Reihe, sehen wir, dass die Rekursion $b_n-3b_{n-1} + b_{n-2}$ gilt.\\
Daraus schließen wir nun, dass $a_n$ die gewünschte Rekursion $x^2-3x+1=0$ von oben erfüllt.\\
Es bleibt also noch zu zeigen, dass sowohl $Fib(2n)$, als auch die Formel $\frac{(3+\sqrt{5})^{n}-(3-\sqrt{5})^{n}}{2^{n}\sqrt{5}}$ dieser Rekursionsvorschrift genügen;\\ 
Das sind aber zwei sehr einfache Rechnungen, die wir uns an dieser Stelle sparen.\\
Damit ist unser Beweis vollständig.\\

\subsection{Rad-Graphen}
Der vorletzte Stop auf unserer Reise sind die sogenannten Wheel-Graphen. Hier wird zu einem zyklischen Graphen $C_n$ mit Knoten $\{v_1,..,v_n\}$, $n \geq 3$ ein weiterer Knoten $z$ hinzugefügt, der mit allen anderen Knoten benachbart ist, sodass der Wheel-Graph $W_{n}$ entsteht (Achtung: $W_n$ hat $n+1$ Knoten).
\begin{Tm}
Für die Anzahl der Spannbäume in einem Rad gilt:
\begin{equation}
 \mathit{k}\left(W_n\right) = \left(\frac{3+\sqrt{5}}{2}\right)^n+\left(\frac{3+\sqrt{5}}{2}\right)^n-2
 \label{wn}
\end{equation}
\end{Tm}
\textbf{Beweis:}\\
Um die Formel für die Berechnung der Anzahl der Spannbäume eines solchen Graphen herzuleiten, lassen wir von ~\cite{sedlacek_1970} inspirieren.
Wir beobachten, dass wir den Fan-Graphen $F_n$ bekommen, wenn wir die Kante $v_1v_n$ aus $W_n$ entfernen.
Die Anzahl der Spannbäume von $F_n$ kennen wir bereits von oben.
Um die Anzahl der Spannbäume von Rädern zu berechnen, zeigen wir zuerst die rekursive Beziehung
\begin{equation}
 \mathit{k}\left(W_{n+1}\right) = \mathit{k}\left(F_{n+1}\right) + \mathit{k}\left(F_n\right) + \mathit{k}\left(W_n\right)
\end{equation}
Um das zu tun, werden die Spannbäume von $W_{n+1}$ in drei verschiedene Klassen einteilen:\\
\par
\begingroup
\leftskip=20pt
\rightskip=20pt
\noindent
\textbf{1)} Alle Spannbäume, die die Kante $v_{n+1}v_1$, aber nicht die Kante $v_{n+1}z$ enthalten; das sind genau so viele, wie die Spannbäume von $W_n$, wie man in der Abbildung~\ref{klasse1} sehen kann. Wir beachten hier, dass ein $W_n$ entsteht, wenn man im $F_n$ eine Kante zwischen den beiden Knoten mit Grad $2$ hinzufügt.\\
\textbf{2)} Alle Spannbäume, die die Kante $v_{n+1}v_1$ nicht enthalten; das sind genau so viele, wie die Spannbäume von $F_{n+1}$; das wird aus Abbildung~\ref{klasse2} ersichtlich\\
\textbf{3)} Alle Spannbäume, die die Kante $v_{n+1}v_1$ und die Kante $v_{n+1}z$ enthalten; Wir beweisen gleich, dass das so viele sind, wie die Spannbäume von $F_n$.\\
\par
\endgroup
%\begin{figure}[h]
    \begin{minipage}{0.45\textwidth}
        \centering
        \includegraphics[width=1\textwidth]{klasse11.png}
        %\caption{In Klasse 1 sind die Spannbäume dieses Graphen, wobei die schwarze Kante in diesen Spannbäumen enthalten sein muss und der grün markierte Teil ein $F_n$ ist.}
        \captionof{figure}{In Klasse 1 sind die Spannbäume dieses Graphen, wobei die schwarze Kante in diesen Spannbäumen enthalten sein muss und der grün markierte Teil ein $F_n$ ist.}
 \label{klasse1} %caption vor label unbedingt
    \end{minipage}\hfill
    \begin{minipage}{0.45\textwidth}
        \centering
        \includegraphics[width=1\textwidth]{klasse22.png}
        %\caption{In Klasse 2 sind genau die Spannbäume dieses Graphen, wobei der grün markierte Teil ein $F_{n+1}$ ist.}
        \captionof{figure}{In Klasse 2 sind genau die Spannbäume dieses Graphen, wobei der grün markierte Teil ein $F_{n+1}$ ist.}
 \label{klasse2} %caption vor label unbedingt
    \end{minipage}
%\end{figure}
\vspace{1cm}

Um zu zeigen dass die Klasse 3 genausoviele Spannbäume enthält wie $F_n$, werden wir beweisen, dass für die Anzahl der Spannbäume in Klasse 3 den gleichen rekursiven Formeln genügen, wie die von $F_n$.\\
Sei $a_n$ die Anzahl der Subgraphen von $F_n$, die aus genau zwei Komponenten bestehen, von denen eine den Knoten $z$ und die andere $v_n$ enthält.\\
Sei $n \geq 2$.
Wir beweisen zuerst, dass $\mathit{k}\left(F_{n+1}\right)=2\mathit{k}\left(F_{n}\right)+a_n$.
Dazu nehmen wir ohne Beschränkung der Allgemeinheit an, dass $v_{n+1}$ nur zu den Knoten $v_n$ und $z$ adjazent ist. Ein Spannbaum des $F_n+1$ ist dann entweder durch hinzufügen einer der beiden Kanten $v_nv_{n+1}$ und $v_{n+1}z$ entstanden, oder durch hinzufügen beider Kanten an einen Subgraphen von $F_n$, der aus genau zwei Komponenten, die Bäume sind, besteht, von denen eine den Knoten $z$ und die andere $v_n$ enthält.\\
Es gilt also $\mathit{k}\left(F_{n+1}\right)=2\mathit{k}\left(F_{n}\right)+a_n$.\\
Wir sehen leicht, dass $a_{n+1}=a_n+\mathit{k}\left(F_n\right)$.\\
Jetzt betrachten wir die Spannbäume aus Klasse 3.\\
Wir definieren dazu $b_n$ als die Anzahl der Spannbäume in Klasse 3, die die Kanten $v_nv_{n+1}$ und $v_{n}z$ nicht enthalten.\\
Sei $M_n$ die Menge der Spannbäume von $W_{n+1}$ aus Klasse 3;\\
Nun zeigen wir, dass $|M_{n}|=2|M_{n-1}|+b_{n-1}$ für $n \geq 4$ ist.\\
Im folgenden betrachten wir nur ein $n \geq 3$. Wir erinnern uns an dieser Stelle daran, dass wir die Knoten beliebig umnummerieren können; in $M_{n}$ sind also die Spannbäume des Graphen aus Abbildung~\ref{mn1}.
\begin{figure}[H]
  \centering
 \includegraphics[width=0.5\textwidth]{mn1.png}
 \caption{}
 \label{mn1} %caption vor label unbedingt
\end{figure}
Wir unterteilen die Spannbäume davon auch in drei Klassen:\\
\par
\begingroup
\leftskip=20pt
\rightskip=20pt
\noindent
Die Erste bilden diejenigen, in denen die Kante $v_{n+1}z$ enthalten ist; Wie wir in der Abbildung~\ref{mn2} erkennen können, sind das soviele, wie in $M_{n-1}$.\\
Die Zweite besteht aus denen, die die Kante $v_nv_{n+1}$ enthalten; In Abbildung~\ref{mn3} sehen wir, dass das auch $|M_{n-1}|$ Stück sind.\\
Die Dritte beinhaltet die, die weder die Kante $v_{n+1}z$ noch die Kante $v_nv_{n+1}$ enthalten; die Anzahl davon ist nach Definition genau $b_{n-1}$.\\
\par
\endgroup
\begin{figure}[H]
    \centering
    \begin{minipage}{0.45\textwidth}
        \centering
        \includegraphics[width=1\textwidth]{mn2.png}
        \caption{}
 \label{mn2} %caption vor label unbedingt
    \end{minipage}\hfill
    \begin{minipage}{0.45\textwidth}
        \centering
        \includegraphics[width=1\textwidth]{mn3.png}
        \caption{}
 \label{mn3} %caption vor label unbedingt
    \end{minipage}
\end{figure}
Daraus schließen wir also $|M_{n}|=2|M_{n-1}|+b_{n-1}$ für $n\geq4$.\\
Wir sehen leicht, dass $\mathit{k}\left(F_2\right) = |M_2|$, $\mathit{k}\left(F_3\right)=|M_3|$ $a_2=b_2=2$ und $a_3=b_3=5$; daraus schließen wir, dass die Anzahl der Spannbäume in Klasse 3 gleich $\mathit{k}\left(F_{n}\right)$ ist, was wir zeigen wollten.\\
Da jeder Spannbaum von $W_{n+1}$ in genau einer der 3 Klassen ist, gilt die rekursive Beziehung
\begin{equation}
\mathit{k}\left(W_{n+1}\right) = \mathit{k}\left(F_{n+1}\right) + \mathit{k}\left(F_n\right) + \mathit{k}\left(W_n\right)
\label{eq:wrek}
\end{equation}
Wir werden nun den Beweis per Induktion über $n \in \mathbb{N}, \, n \geq 3$ vervollständigen, wobei uns natürlich zu Gute kommt, dass uns die Anzahl der Spannbäume von Fan-Graphen schon bekannt ist.\\
Für unseren Induktionsanfang sehen wir -zum Beispiel durch Anwendung von Kirchhoffs Matrix-Tree-Theorem- leicht, dass \begin{equation}
\mathit{k}\left(W_3\right) = 16 = \left(\frac{3+\sqrt{5}}{2}\right)^3+\left(\frac{3+\sqrt{5}}{2}\right)^3-2.
\end{equation}
Wir nehmen nun an, dass für ein $n \in \mathbb{N}$ die Formel 
\begin{equation}
 \mathit{k}\left(W_n\right) = \left(\frac{3+\sqrt{5}}{2}\right)^n+\left(\frac{3+\sqrt{5}}{2}\right)^n-2
\end{equation}
gilt.\\
Damit bleibt noch zu zeigen, dass
\begin{equation}
 \mathit{k}\left(W_{n+1}\right) = \left(\frac{3+\sqrt{5}}{2}\right)^{n+1}+\left(\frac{3+\sqrt{5}}{2}\right)^{n+1}-2.
\end{equation}
Das werden wir nun einfach ausrechnen.
Nachdem wir im vorherigen Kapitel herausgefunden haben, wieviele Spannbäume Fan-Graphen haben, setzen wir das und unsere Induktionsannahme in die Gleichung (\ref{eq:wrek}) ein, und erhalten:\\
\begin{equation}
\begin{aligned}
\mathit{k}\left(W_{n+1}\right) ={} & \frac{\left(3+\sqrt{5}\right)^{n+1}-\left(3-\sqrt{5}\right)^{n+1}}{2^{n+1}\sqrt{5}} + \frac{\left(3+\sqrt{5}\right)^{n}-\left(3-\sqrt{5}\right)^{n}}{2^{n}\sqrt{5}}\\
& + \left(\frac{3+\sqrt{5}}{2}\right)^n+\left(\frac{3-\sqrt{5}}{2}\right)^n-2
\end{aligned}
\end{equation}
Wir bringen fast alles auf einen Nenner, sortieren die Terme und bekommen
\begin{equation}
\begin{aligned}
\mathit{k}\left(W_{n+1}\right) = {}  & \frac{\left(3+\sqrt{5}+2+2\sqrt{5}\right)\left(3+\sqrt{5}\right)^{n}}{2^{n+1}\sqrt{5}} \\%% {} steht da nur, weils hin muss, wegen dem =
                        & -\frac{\left(3+\sqrt{5}+2-2\sqrt{5}\right)\left(3-\sqrt{5}\right)^{n}}{2^{n+1}\sqrt{5}}-2 
\end{aligned}
\end{equation}
\todo[inline]{evtl. zusammengehörige Terme evtl. farbig markieren}
Ausrechnen führt uns zu\\
\begin{equation}
\mathit{k}\left(W_{n+1}\right) = \left(\frac{3+\sqrt{5}}{2}\right)^{n+1}+\left(\frac{3+\sqrt{5}}{2}\right)^{n+1}-2
\end{equation}
Damit ist unser Induktionsbeweis abgeschlossen und wir haben gezeigt, dass unser Satz \ref{wn} über die Anzahl der Spannbäume in einem Rad gilt.
\begin{flushright} $\Box$ \end{flushright} 

\graphicspath{{grafiken/}}

\subsection{circulant Graphs}
%Als Letztes werden wir Graphen betrachten, deren Adjazenzmatrizen zyklisch sind.\\
%%Wo treten die auf (z.B. Circulant Graphs sind Cayley-Graphen zyklischer Gruppen)
Wir nennen einen Graphen circulant mit $n$ Knoten, wenn für $n \in \mathbb{N}$ und eine Menge $I \subset{\{1,..,\lfloor \frac{n}{2} \rfloor \}}\subset{\mathbb{N}}$ gilt, dass jeder Knoten $v$ genau zu jedem Knoten $(v+i) (\mod{n})$ mit $i \in I$ benachbart ist; wir bezeichnen solch einen Graphen kurz mit $C_n^I$.\\
\todo[inline, color=blue]{Im folgenden Satz nachm Leerzeichen einfügen dieselben zwischen nxn und Matrix wieder wegmachen, das gehört nämlich zusammen}
Wir erinnern uns, dass eine $n\times n$-Matrix zyklisch genannt wird, falls jede Spalte aus der vorherigen durch Anwendung der Permutation $(1...n)$ hervorgeht.
Das ist bei den Adjazenzmatrizen unserer circulant Graphs, aufgrund dessen, wann Konten benachbart sind, natürlich der Fall.
Zu Gute kommt uns das bei der Berechnung der Anzahl von Spannbäumen in circulant Graphs, denn die Eigenwerte einer zyklischen Matrix sind wohlbekannt.%%gibts das Wort überhaupt?
Um die Formel für die Anzahl der Spannbäume überhaupt zu verstehen, müssen wir einen weiteren Begriff einführen.%%Grad von C_n^I erklären, (Grad maximalgrad eines knoten?)
Nachdem wir nun alles beisammen haben, formulieren wir folgenden Satz:

\begin{Tms}
Für die Anzahl der Spannbäume in circulant Graphs von Grad d gilt:\\
\begin{equation}
\mathit{k}\left( C_n^I \right) = \frac{1}{n} \prod_{j=1}^{n-1} \left(4 \sum_{i \in I} \sin^2 \left( \frac{ij\pi}{n}\right) \right),\,falls\,d\,gerade\,ist
\end{equation}
\begin{equation}
\mathit{k}\left( C_n^I \right) = \frac{1}{n} \prod_{j=1}^{n-1} \left(4 \sum_{i \in I} \sin^2 \left( \frac{ij\pi}{n}\right)-(-1)^j+1\right),\,falls\,d\,ungerade\,ist
\end{equation}
\end{Tms}

\textbf{Beweis:}\\
Wir beweisen den Satz wie ~\cite{wang_yang_1984}.\\
\todo[inline]{Beweis schreiben}

\begin{Bsps}[$C_n^2$ - Das Quadrat eines Kreises]
\todo[inline, color=green]{Bild von einem Square of a cycle}
\todo[inline]{Herleitung Formel}
%Im Notfall einen anderen Beweis für $C_n^{(1,2)}$ raussuchen (eher nicht), der jetzige ist lang, allerdings nutzt er er das MTT was von Vorteil ist (ich werde die Rechnungen verkürzen und nur die wichtigen Teile ausführlich machen, sonst werden das 4-6 Seiten quasi nur mit Rechnungen)
\end{Bsps} 


%\graphicspath{{grafiken/zusammenfassung/}}

\chapter{Zusammenfassung}

Das in Abb.\,\ref{fig:LoremIpsum} beschriebene Verhalten charakterisiert die vorliegende Arbeit. Au\ss erdem sei an dieser Stelle auf Kap.\,\ref{chap:einleitung} verwiesen.

\begin{figure}[htbp]
\centering
\includegraphics[width=0.5\textwidth]{LoremIpsum}
\caption{Beschreibung des oben stehenden Bildes.}
\label{fig:LoremIpsum}
\end{figure}

An dieser Stelle will ich noch \cite{sahin:penning} zitieren, au\ss erdem ist in \cite{zibell:diplarbeit} gezeigt, dass es regnet \cite{zibell:phd}.
\cleardoublepage

\bibliography{bachelorarbeit}{}
\bibliographystyle{plain}

%\begin{appendix}
%\include{anhang}

%\end{appendix}

\clearpage{\pagestyle{empty}\cleardoublepage}
%\cleardoublepage
\thispagestyle{empty}

\vspace*{1cm}
{\huge \textbf{Selbständigkeitserklärung}}\\
\vspace*{1.5cm}

Ich versichere hiermit, die vorliegende Arbeit mit dem Titel

\begin{center}
	\textbf{Das Matrix-Tree-Theorem}
\end{center}

selbständig verfasst zu haben und keine anderen als die angegebenen Quellen und Hilfsmittel verwendet zu haben.

\vspace*{3cm}

Christopher Mann

\vspace*{1cm}
München, den \myformat\today

\end{document}
