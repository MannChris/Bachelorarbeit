\subsection{$W_n$ (Räder)}
Der vorletzte Stop auf unserer Reise sind die sogenannten Wheel-Graphen. Hier wird zu einem zyklischen Graphen $C_n$ mit Knoten $\{v_1,..,v_n\}$, $n \geq 3$ ein weiterer Knoten $z$ hinzugefügt, der mit allen anderen Knoten benachbart ist, sodass der Wheel-Graph $W_{n}$ entsteht (Achtung: $W_n$ hat $n+1$ Knoten).\\
%V1:
%Um eine Formel für die Berechnung der Anzahl der Spannbäume eines solchen Graphen herzuleiten, werden wir vorgehen wie in ~\cite{sedlacek_1970}.
%Wie nehmen uns also einen leicht modifizierten Graphen $H_n$ vor; dabei entfernen wir die Kante $v_1v_n$ aus $W_{n}$. Des weiteren definieren wir $H_1 := K_2$ und $H_2 := K_3$. 
%Zuerst werden wir die Anzahl der Spannbäume von $H_n$ berechnen und anschließend damit die von $W_n$.
%Wier zeigen also zunächst, dass die Anzahl der Spannbäume von $H_n$ gleich: $\frac{(3+\sqrt{5})^n - (3-\sqrt{5})^n}{2^n \sqrt{5}}$ ist.
%Dazu sei $a_n$ die Anzahl der Spannbäume von $H_n$ und sei $b_n$ die Anzahl der Subgraphen des $H_n$ die alle Knoten von $H_n$ enthalten und aus zwei Komponenten bestehen, von denen die eine $v_n$ und die andere $z$ enthält. Mithilfe von vollständiger Induktion zeigen wir, dass $a_1 = b_1 = 1$ und die folgenden rekursiven Beziehungen gelten:
%$a_{n+1} = 2a_n + b_n$ und $b_{n+1} = a_n + b_n$
%Es ist leicht zu sehen, dass $a_1 = b_1 = 1$ und $a_2 = 3$, sowie $b_2 = 2$
%%Induktionsbeweis...
%V2:\\
\begin{Tm}
Für die Anzahl der Spannbäume in einem Rad gilt:
\begin{equation}
 \mathit{k}(W_n) = (\frac{3+\sqrt{5}}{2})^n+(\frac{3+\sqrt{5}}{2})^n-2
 % (1) erstes ^n fehlte vorher, weil das in der Quelle falsch drinsteht
\end{equation}
\end{Tm}

\textbf{Beweis:}\\
Um die Formel für die Berechnung der Anzahl der Spannbäume eines solchen Graphen herzuleiten, lassen wir von ~\cite{sedlacek_1970} inspirieren.
Wir beobachten, dass wir den Fan-Graphen $F_n$ bekommen, wenn wir die Kante $v_1v_n$ aus $W_n$ entfernen.
Die Anzahl der Spannbäume von $F_n$ kennen wir bereits von oben.
%nicht notwendig (1)
%Wir werden zeigen, dass $\mathit{k}(W_n) = \mathit{k}(F_n) + 2 \sum_{j=2}^n\mathit{k}(F_{j-1})$;
%damit können wir danach die Anzahl der Spannbäume von $W_n$ berechnen.
%Als ersten Schritt dahin beweisen wir, dass für $n \geq 3$ die nachfolgende rekursive Beziehung gilt:
Um die Anzahl der Spannbäume von Rädern zu berechnen, zeigen wir folgende rekursive Beziehung
\begin{equation}
 \mathit{k}(W_{n+1}) = \mathit{k}(F_{n+1}) + \mathit{k}(F_n) + \mathit{k}(W_n)
\end{equation}
Um das zu tun, werden die Spannbäume von $W_{n+1}$ in drei verschiedene Klassen einteilen, wie man auch in den Abbildungen unten sehen kann:\\%%?Klassen hier richtiger Begriff?
%Die wegkommentierten stimmen zwar theoretisch, ist aber nicht zielführend
%1) Alle Spannbäume, die die Kante $v_1v_{n+1}$ nicht enthalten; das sind genau die Spannbäume von $F_{n+1}$.\\
%2) Alle Spannbäume, die die Kante $v_1v_{n+1}$ enthalten, jedoch nicht die Kante $v_1z$; das sind die Spannbäume des Graphen $W_{n+1} \slash v_1v_{n+1}$, den wir durch Kontraktion der Kante $v_1v_{n+1}$ aus $W_{n+1}$ erhalten - dieser Graph ist aber $W_n$.\\
%3) Alle Spannbäume, die die Kanten $v_1v_{n+1}$ und $v_1z$ beinhalten; das sind die Spannbäume des 
%Graphen, den wir durch die Kontraktion der Kante $v_1z$ gewinnen, also von $F_n$, wie wir aus der nachfolgenden Grafik entnehmen können.\\
1) Alle Spannbäume, die die Kante $v_{n+1}v_1$, aber nicht die Kante $v_{n+1}z$ enthalten; das sind genau so viele, wie die Spannbäume von $W_n$. \\%%Grafik dazu einfügen
2)Alle Spannbäume, die die Kante $v_{n+1}v_1$ nicht enthalten; das sind genau so viele, wie die Spannbäume von $F_{n+1}$.\\%%Grafik dazu einfügen
3) Alle Spannbäume, die die Kante $v_{n+1}v_1$ und die Kante $v_{n+1}z$ enthalten; jetzt beweisen wir, dass das so viele sind, wie die Spannbäume von $F_n$.\\
Dafür werden wir zeigen, dass für die Anzahl der Spannbäume in Klasse $3$ den gleichen rekursiven Formeln genügen wie die von $F_n$.\\
%%Grafik dazu einfügen
Wie wir sehr leicht sehen können ist jeder Spannbaum von $W_{n+1}$ in genau einer dieser Klassen, also gilt die rekursive Beziehung
$\mathit{k}(W_{n+1}) = \mathit{k}(F_{n+1}) + \mathit{k}(F_n) + \mathit{k}(W_n)$\\
%nicht notwendig(1)
%Unsere Formel %%Formel von oben
%lässt sich - zum Beispiel durch vollständige Induktion über $n \in \mathbb{N}$ - sehr einfach
%verifizieren.\\
Wir werden nun den Beweis per Induktion über $n \in \mathbb{N}, \, n \geq 3$ vervollständigen, wobei uns natürlich zu Gute kommt, dass uns die Anzahl der Spannbäume von Fan-Graphen schon bekannt ist.\\
Für unseren Induktionsanfang sehen wir -zum Beispiel durch Anwendung von Krichhoffs Matrix-Tree-Theorem- leicht, dass $\mathit{k}(W_3) = 16 = (\frac{3+\sqrt{5}}{2})^3+(\frac{3+\sqrt{5}}{2})^3-2$.\\
Wir nehmen nun an, dass für ein $n \in \mathbb{N}$ die Formel $\mathit{k}(W_n) = (\frac{3+\sqrt{5}}{2})^n+(\frac{3+\sqrt{5}}{2})^n-2$ gilt.\\
Damit bleibt noch zu Zeigen, dass  $\mathit{k}(W_{n+1}) = (\frac{3+\sqrt{5}}{2})^{n+1}+(\frac{3+\sqrt{5}}{2})^{n+1}-2$.\\
Wie wir oben bereits gezeigt haben, gilt
$\mathit{k}(W_{n+1}) = \mathit{k}(F_{n+1}) + \mathit{k}(F_n) + \mathit{k}(W_n)$\\
Nachdem wir im vorherigen Kapitel herausgefunden haben, wieviele Spannbäume Fan-Graphen haben, setzen wir das und unsere Induktionsannahme sofort ein, und erhalten:\\
$\mathit{k}(W_{n+1}) = \frac{(3+\sqrt{5})^{n+1}-(3-\sqrt{5})^{n+1}}{2^{n+1}\sqrt{5}} + \frac{(3+\sqrt{5})^{n}-(3-\sqrt{5})^{n}}{2^{n}\sqrt{5}} + (\frac{3+\sqrt{5}}{2})^n+(\frac{3-\sqrt{5}}{2})^n-2$\\%%fehler ausgebessert (2)
Einfaches Umformen führt uns zu\\

%%Formel konkret berechnen
Damit haben wir erfolgreich gezeigt, dass für die Anzahl der Spannbäume in $W_n$ gilt:\\

%% Alle Rechnungen evtl. in equations packen

