\subsection{Warm-up}
Als kleines Aufwärmprogramm für den Rest dieser Arbeit werden wir in diesem Kapitel die Anzahl der Spannbäume von ein paar wohlbekannten, sehr einfachen Graphen mit Kirchhoffs Matrix-Tree-Theorem berechnen.\\
Unsere Ergebnisse aus diesem Kapitel wollen wir uns im Kapitel zu kartesischen Produkten von Graphen zuuntze machen. Weil wir die Eigenwerte der Laplacematrizen im Kapitel über kartesische Produkte von Graphen brauchen, werden wir diese hier jedes Mal ausrechnen.\\
\begin{Lms}
 Der Pfad-Graph $P_n$ mit $n$ Knoten hat genau einen Spannbaum.
\end{Lms}
\textbf{Beweis:}\\
Dass ein Pfad-Gaph nur einen Spannbaum hat ist offensichtlich; er ist selbst schon ein Baum.\\
Wir sind aber auch an den Eigenwerten der Laplacematrix interessiert.\\
\begin{equation}
L(P_n)=
\begin{pmatrix}
1&-1&0&\ldots&\ldots&\ldots\\
-1&2&-1&0&\ldots&\ldots\\
0&\ldots&\ldots&\ldots&\ldots&\ldots\\
\ldots&\ldots&\ldots&\ldots&\ldots&0\\
\ldots&\ldots&0&-1&2&-1\\
\ldots&\ldots&\ldots&0&-1&1\\
\end{pmatrix}
\end{equation}
Lemma 2 von~\cite{daoud_2014}, zeigt unter Verwendung einiger Eigenschaften von Chebychev-Polynomen, dass die Eigenwerte $\neq 0$ dieser Matrix gleich $2-2\cos \left(\frac{\pi k}{n}\right)$ für $k \in \{1,\ldots,n-1\}$ sind.
Mit Kirchhoffs Matrix-Tree-Theorem folgt
\begin{equation}
 \mathit{k}(P_n)=\frac{1}{n}\prod_{j=1}^{n-1} \left(2-2\cos \left(\frac{\pi j}{n}\right)\right)
\end{equation}
Wir rechnen weiter
\begin{equation}
\begin{split}
 \mathit{k}(P_n)={} & \frac{2^{n-1}}{n}\prod_{j=1}^{n-1} \left(1-\cos \left(\frac{\pi j}{n}\right)\right) \\
  ={}& \frac{2^{n-1}}{n}\prod_{j=1}^{n-1} \left(\left(1-\cos \left(\frac{\pi j}{n}\right)\right)^2\right)^{\frac{1}{2}} \\
  ={}&\frac{2^{n-1}}{n}\prod_{j=1}^{n-1} \left(\left(1-2\cos^2 \left(\frac{\pi j}{n}\right)+\cos^2 \left(\frac{\pi j}{n}\right)\right)\right)^{\frac{1}{2}} \\
  ={}& \frac{2^{n-1}}{n}\prod_{j=1}^{n-1} \left(\sin\left(\frac{\pi k}{n}\right) \right)
  \end{split}
\end{equation}
An dieser Stelle verweisen wir auf~\cite{fiktor_2010} um folgendes zu zeigen:
\begin{equation}
 \prod_{j=1}^{n-1} \left(\sin\left(\frac{\pi k}{n}\right) ist \right)=n2^{n-1}
 \label{fiktor}
\end{equation}
Damit folgt, dass $P_n$ wie erwartet einen einzigen Spannbaum besitzt.
\begin{flushright} $\Box$ \end{flushright}
Die zweite Klasse von Graphen, die wir in diesem Kapitel betrachten, sind Kreis-Graphen.
Sie sind der einfachste Spezialfall von zirkulären Graphen, die wir später auch noch behandeln werden. In einem Kreis-Graphen $C_n$ mit $n$ Knoten sind genau die aufeinanderfolgenden Knoten, sowie der erste und der letzte adjazent. 
\begin{Lms}
 Der Kreis-Graph $C_n$ mit $n$ Knoten hat genau $n$ Spannbäume.
\end{Lms}
\textbf{Beweis:}
\begin{equation}
L(C_n)=
\begin{pmatrix}
1&-1&0&\ldots&\ldots&\ldots\\
-1&2&-1&0&\ldots&\ldots\\
0&\ldots&\ldots&\ldots&\ldots&\ldots\\
\ldots&\ldots&\ldots&\ldots&\ldots&0\\
\ldots&\ldots&0&-1&2&-1\\
\ldots&\ldots&\ldots&0&-1&1\\
\end{pmatrix}
\end{equation}
Lemma 3, wieder aus~\cite{daoud_2014} bestimmt die Eigenwerte $\lambda_k,\; k \in \{0,\ldots,n-1\}$
dieser Matrix als $\lambda_k = 2-2\cos {\left(\frac{2\pi k}{n}\right)}$.
Um das weiter zu vereinfachen, werden wir jetzt unser Wissen über trigonometrische Funktionen verwenden.
\begin{equation}
\begin{split}
 \lambda_k={} & 2 \left(1-\cos \left(\frac{2\pi j}{n}\right)\right)\\
 = {}&2 \left( 1 - \cos^2\left(\frac{\pi k}{n}\right)+\sin^2\left(\frac{\pi k}{n}\right) \right)\\
 = {}& 4 \left(\sin^2\left(\frac{\pi k}{n}\right) \right)\\
\end{split}
\label{ewc}
\end{equation}
Mit Kirchhoffs Matrix-Tree-Theorem folgt
\begin{equation}
 \mathit{k}(C_n)=\frac{4^{n-1}}{n}\prod_{j=1}^{n-1} \left(\sin^2\left(\frac{\pi k}{n}\right) \right)\\
\end{equation}
%\begin{equation}
 %\mathit{k}(C_n)=\frac{1}{n}\prod_{j=1}^{n-1} \left(2-2\cos \left(\frac{2\pi j}{n}\right)\right)
%\end{equation}
%Jetzt werden wir unser Wissen über trigonometrische Funktionen verwenden.
%\begin{equation}
%\begin{split}
 %\mathit{k}(C_n)={} & \frac{2^{n-1}}{n}\prod_{j=1}^{n-1} \left(1-\cos \left(\frac{2\pi j}{n}\right)\right)\\
 %= {}&\frac{2^{n-1}}{n}\prod_{j=1}^{n-1} \left( 1 - \cos^2\left(\frac{\pi k}{n}\right)+\sin^2\left(\frac{\pi k}{n}\right) \right)\\
 %= {}& \frac{4^{n-1}}{n}\prod_{j=1}^{n-1} \left(\sin^2\left(\frac{\pi k}{n}\right) \right)\\
%\end{split}
%\end{equation}
Wir verwenden die Gleichung (\ref{fiktor}) wieder und schließen $\mathit{k}(C_n)=n$.
\begin{flushright} $\Box$ \end{flushright} 
