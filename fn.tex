\subsection{$F_n$ (Fan)(Fächer?)}
Nun werden wir Fan-Graphen $F_n$, für $n \geq 1$betrachten. Diese entstehen wenn wir an einen Pfad-Graphen $P_{n}$ einen weiteren Knoten so ankleben, dass er mit allen übrigen Knoten adjazent ist. 
% Interessanterweise treffen wir hier auf die Fibonaccizahlen, blablabla
\todo[inline, color= red]{Achtung: Fn bei uns ist Fn+1 im Paper}
Wir wollen in diesem Kapital folgendes über die Anzahl der Spannbäume in Fan-Graphen zeigen:
\begin{Tms}
 %%Formel hier
 \label{ThmFn}
\end{Tms}
\textbf{Beweis:}
Diesmal halten wir uns an einen Beweis von Bogdanowicz ~\cite{bogdanowicz_2008}, wobei dieser $F_n$ leicht anders definiert.\\
Zuerst werden wir zeigen, dass ein Kofaktor der Laplacematrix von Fan-Graphen einer bestimmten Rekursion folgt und dann, dass $Fib(2n)$ der die gleiche Rekursionsvorschrift einhält; Mit Kirchhoffs Matrix-Tree-Theorem folgt dann der Satz.\\
Wir betrachten also zunächst die Laplacematrix von $F_n$; wir dürfen dazu die Knoten nummerieren wie wir wollen, also bekommen wir\\
\todo[inline]{matrix, zeilenumbrüche checken}
Wir brauchen einen beliebigen Kofaktor davon, deshalb streichen wir die letzte Zeile und Spalte und erhalten\\
\todo[inline]{matrix, zeilenumbrüche checken, An nennen}
Die Determinante dieser Matrix ist der gesuchte Kofaktor; wir benennen sie mit $a_n$.\\
Nun zeigen wir,dass die Folge $(a_n)_{n \in \mathbb{N}}$ der Rekursion $x^2-3x+1=0$ folgt, \\wobei $x$ den Shift-Operator $a_n = xa_{n-1}$ darstellt. \\
Wir entwickeln $A_n$ nach der ersten Reihe und erhalten $a_n = 2b_{n-1} - b_{n-2}$, wobei $b_i$ die Determinante der folgenden Hilfsmatrix ist:\\
\todo[inline]{matrix, zeilenumbrüche checken}
Entwickeln wir die Determinante dieser Matrix für $i=n$ ebenfalls nach der ersten Reihe, sehen wir, dass die Rekursion $b_n-3b_{n-1} + b_{n-2}$ gilt.\\
Daraus schließen wir nun, dass $a_n$ die gewünschte Rekursion $x^2-3x+1=0$ von oben erfüllt.\\
Es bleibt also noch zu zeigen, dass sowohl $Fib(2n)$, als auch die Formel $\frac{(3+\sqrt{5})^{n}-(3-\sqrt{5})^{n}}{2^{n}\sqrt{5}}$ dieser Rekursionsvorschrift genügen;\\ 
Das sind aber zwei sehr einfache Rechnungen, die wir uns an dieser Stelle sparen.\\
Damit ist unser Beweis vollständig.\\
